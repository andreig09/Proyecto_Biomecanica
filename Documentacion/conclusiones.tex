\chapter{Conclusiones}
\label{conclusiones}

Se obtuvo en forma íntegra un sistema óptico de captura de movimiento basado en marcadores, que a partir de las capturas de video de una persona en un ambiente de laboratorio con las condiciones adecuadas, obtiene la posición 3D de  los marcadores presentes en el cuerpo de dicha persona, logrando representar su movimiento con una precisión del orden del centímetro.\\


Si bien inicialmente el objetivo era que el sistema funcionara por lo menos para el caso de uso de la marcha, se han probado otros  movimientos con resultados aceptables. La aplicación desarrollada permite a partir de múltiples capturas de vídeo de un sujeto en movimiento, detectar los marcadores en cada toma de video. Junto a la información de las cámaras, luego reconstruye la posición de los marcadores en el espacio y finalmente logra identificar cada marcador a lo largo de la secuencia temporal. Cabe destacar que el sistema implementado no es solo óptico, sino que es lo bastante general para funcionar con cualquier sistema de adquisición que genere imágenes, por ejemplo con las imágenes infrarrojas de un sistema Vicon \cite{vicon}. \\ 


La revisión bibliográfica inicial permite encontrar los procesos principales que normalmente conforman un sistema de captura de movimiento: \emph{Adquisición}, \emph{Calibración}, \emph{Detección de marcadores}, \emph{Reconstrucción} y \emph{Seguimiento};  obteniendo a su vez una idea del estado del arte de dichos procesos. En dicha etapa se realiza una clasificación de los documentos de acuerdo a la relevancia que prestan y se definen los métodos que posteriormente se utilizaron para implementar los distintos procesos que componen el sistema. \\  

Al relevar las aplicaciones existentes, no se encuentra software de código abierto que se adapte a las características necesarias para utilizar como base sobre la cual montar este proyecto. Por esto se decide implementar los distintos procesos por cuenta propia, teniendo esta etapa por momentos un carácter más de investigación científica que de proyecto ingenieril.\\

Se efectúa una búsqueda de secuencias de movimiento sobre las cuales desarrollar el sistema, si bien se encontraron numerosas bases de datos, las mismas terminan siendo descartadas por no ajustarse completamente  a las hipótesis que se plantearon en este trabajo. De todas maneras cabe destacar que se genera un relevamiento de bases de datos para el movimiento humano y se profundiza en las características usuales presentes en dichas bases de datos, logrando obtener un conjunto de conceptos y herramientas importantes para el proyecto. Previo análisis de los parámetros involucrados, se enumeran consideraciones a tener en cuenta a la hora de generar un laboratorio de captura.
Con ayuda de la suite de animación 3D \emph{Blender} y utilizando fuentes BVH de captura de movimiento disponibles en las distintas bases de datos relevadas, se obtiene un laboratorio de captura de movimiento virtual que permite generar secuencias sintéticas de movimiento con sus respectivos videos. Esto último permite contar con un ground truth sobre el cual validar los algoritmos en cada etapa del sistema así como obtener un benchmark para medir el rendimiento de los mismos. Cabe destacar que el sistema de prueba realizado, permite probar futuros algoritmos.
Se implementa una estructura de datos que soporta la información de interés tanto de secuencias sintéticas como reales y un prototipo de base de datos, lo suficientemente general y útil para trabajar con sistemas de captura de movimiento basada en marcadores.\\ 


La detección de marcadores en tomas individuales (para cada cámara) se logra mediante segmentación y filtros de objetos de complejidad baja. El error medido para calcular los centros de los marcadores según nuestros requerimientos es bajo, por lo que se considera exitosa la implementación del bloque para el caso sintético. Para el caso real los resultados no fueron tan alentadores, sobre todo porque las secuencias reales que se consiguieron no poseen condiciones de laboratorio favorables para el procesamiento óptico y no se encuentran completamente en las hipótesis planteadas en este proyecto. Sin embargo se pudo realizar una detección aceptable trabajando en conjunto con un extractor de fondo. El ajuste de los parámetros internos a la segmentación permite variar los resultados para poder descartar figuras extrañas a los marcadores que se quieren detectar.\\ 

En la etapa de calibración se ha desarrollado un algoritmo que permite realizar la calibración de cámaras a partir de la metodología de captura utilizada por un grupo de investigadores y médicos del Hospital de Clínicas. Por otra parte, de manera independiente, se han analizado dos implementaciones (toolbox) existentes como posibles métodos para calibrar un conjunto de cámaras de un laboratorio de captura de movimiento. Dichos métodos se probaron sobre el laboratorio virtual implementado en el entorno \emph{Blender}. Se concluye que uno de estos toolbox, el \textit{Multi-Camera Self-Calibration Toolbox} \cite{amcctoolbox}, presenta mayor flexibilidad de uso en los laboratorios de captura que utilicen un número elevado de cámaras. Se plantea como trabajo futuro realizar una serie de pruebas de \textit{performance} sobre dichas implementaciones y medir sus impactos en las restantes etapas del sistema.\\

La reconstrucción permite generar los puntos en el espacio siempre que se tenga información para emparejar puntos en dos vistas y confirmar la relación. En las condiciones adecuadas, la reconstrucción cumple con su cometido de generar los puntos con errores del orden del centímetro, lo cual se considera aceptable para un sistema de captura de movimiento de estas características. 
Las secuencias reales disponibles inicialmente no tuvieron una performance aceptable, por lo que se tuvo que re-formular el algoritmo realizado inicialmente para las secuencias sintéticas. Si bien los problemas de detección de marcadores y calibración influyen considerablemente en la performance de la reconstrucción, un análisis posterior y pruebas realizadas en secuencias sintéticas que simulaban el caso real, revelaron que el número de cámaras cambia significativamente el problema, volviendo obsoleto el algoritmo inicialmente propuesto cuando se trabaja con menos de 6 cámaras. Por lo que se implementa un nuevo algoritmo que permite trabajar con las 3 cámaras de la secuencia real.
Finalmente en conjunto ambos algoritmos solucionan los problemas de reconstrucción al variar el número de cámaras, permitiendo trabajar con capturas de al menos tres cámaras de manera aceptable. 
Actualmente los algoritmos de reconstrucción propuestos son generales y no introducen restricciones sobre los marcadores. Se propone utilizar en futuras revisiones, la relación presente entre marcadores adyacentes en capturas de movimiento de personas, con el fin de incrementar la performance. 
\\ 





Se implementó un algoritmo de seguimiento basado en restricciones de velocidad y trayectorias, asumiendo la hipótesis de correcto muestreo de la secuencia y el buen desempeño del proceso de reconstrucción. Si las pérdidas son puntuales y no se mantienen en el tiempo, se implementaron medidas para recuperar trayectorias continuando el movimiento. Los resultados de performance de seguimiento son los mismos que los de reconstrucción pero se pueden medir los errores de manera más específica, teniendo para cada marcador el mismo rendimiento satisfactorio que para el conjunto entero y toda posible discontinuidad puede ser detectada y reparada para mantener la continuidad tanto de velocidad como de aceleración. 
\\ 


Es importante tener presente los pasos que restan para tener una base de datos con secuencias reales.
En este trabajo se exploran en detalle distintos factores a tomar en cuenta  a la hora de armar un laboratorio de captura y se efectúan una serie de recomendaciones o definiciones, tanto para su armado como para la realización de capturas en condiciones óptimas para un posterior procesamiento. También se define una estructura que almacena toda la información recabada en dichas capturas y permite trabajar a lo largo de los diferentes procesos que componen el sistema de procesamiento. Por lo que para obtener una base de datos real con secuencias ópticas, solo resta efectuar las capturas de acuerdo a las condiciones que se definen y calibrar las cámaras. 
Con estas nuevas secuencias y una posterior evaluación se pueden generar mejoras en los algoritmos del sistema.\\


Cabe destacar que se está aumentando la reproducibilidad en el área. Pues se cuenta con un sistema completo y estructurado donde fácilmente se pueden modificar, ingresar y probar distintas partes, comparando sus desempeños con las métricas ya definidas en el Capítulo \ref{evaluacion}.\\ 



El sistema actual devuelve toda la información de los marcadores relevante para el usuario final en una matriz de Matlab, esto no es un problema para los investigadores en Biomecánica, pues los mismos efectúan habitualmente sus cálculos en dicho programa. Fácilmente se pueden efectuar cálculos de centro de masa, velocidades y aceleraciones, tanto parciales como totales.
También se actualiza y devuelve toda la información generada a lo largo del procesamiento en estructuras de Matlab así como en archivos .xml. Si bien se desarrollaron herramientas de visualización de secuencias de movimiento en Matlab, las mismas son básicas. Por lo que en futuros trabajos se pueden generalizar sus prestaciones de manera más acorde al especialista. Otra alternativa con bastante potencial es exportar la información de salida del sistema nuevamente al entorno Blender, de manera tal que el especialista realice un análisis más interactivo sobre un modelo virtual.\\ 



Como trabajo a futuro queda explorar las opciones para robustecer el sistema en condiciones más exigentes asociadas al caso real, por otro lado implementar las medidas adicionales con restricciones para mantener coherencia asociada a un modelo físico, ampliar la interfaz gráfica para cada módulo, haciendo la misma más amigable al usuario y/o orientada al experto (calibración interactiva, bloque de visualización de resultados, entre otros).
Si bien esta primera versión del sistema no está a la altura de los sistemas comerciales relevados en este documento, se logra dar el primer paso, sentando las bases y condiciones necesarias para poder continuar con el proyecto.
El potencial actual tanto del sistema como de la base de datos y su posible expansión permiten afirmar que se generaron herramientas a tener en cuenta en futuros proyectos de captura de movimiento.
