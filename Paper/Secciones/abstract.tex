
\begin{abstract}
%\boldmath
En este artículo se propone un sistema óptico de captura de movimiento basado en marcadores para facilitar la tarea en el análisis biomecánico del movimiento de las personas.

%La propuesta inicial fue realizada por investigadores de biomecánica del Departamento de Biofísica de la Facultad de Medicina de la Universidad de la República, Uruguay, en busca de una herramienta de código abierto que le permita obtener datos y estadísticas específicas que las herramientas existentes no pueden ofrecer.

Se elabora una aplicación con los bloques fundamentales que componen un sistema de estas características, utilizando los lenguajes \emph{C/C++}, \emph{Python} y \emph{Matlab}. Estos bloques son independientes unos de otros, lo que da la posibilidad de modificarlos o sustituirlos sin afectar el resto del sistema.

También se crea un prototipo de base de datos, con secuencias de videos sintéticas, y un conjunto de algoritmos para medir la performance de cada bloque y del sistema en su totalidad.

Las pruebas realizadas sobre el software implementado reflejaron que el mismo tiene una precisión del orden del centímetro. Estos resultados son buenos para ser una primera versión y teniendo en cuenta que los algoritmos utilizados en cada bloque son de complejidad baja y se pueden optimizar en todos sus aspectos.

\end{abstract}

% IEEEtran.cls defaults to using nonbold math in the Abstract.
% This preserves the distinction between vectors and scalars. However,
% if the journal you are submitting to favors bold math in the abstract,
% then you can use LaTeX's standard command \boldmath at the very start
% of the abstract to achieve this. Many IEEE journals frown on math
% in the abstract anyway.

% Note that keywords are not normally used for peerreview papers.
\begin{IEEEkeywords}
Biomecánica, calibración, detección de marcadores, reconstrucción, seguimiento.
\end{IEEEkeywords}